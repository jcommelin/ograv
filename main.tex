\documentclass{article}
\usepackage{amsmath,amssymb}

\usepackage{microtype}
\usepackage{geometry}
\usepackage{biblatex}
\bibliography{\jobname.bib}

\def\HH{\textnormal{H}}

\title{On ordinary good reduction of abelian varieties}
\author{Johan Commelin}

\begin{document}
\maketitle

\section{General plan}

Give an overview of what is known:
\cite{MR1603865},
\cite{MR3494322},
Noot?

Prove that having ordinary reduction at~$v$ only depends on the adjoint motive.
Conclude that it suffices to look at irreducbile abelian motives.

Prove the good ordinary reduction conjecture
for irreducible abelian motives of type $A_n^\wedge$.

$\bullet$\quad Need to figure out how to use categories of motives
with coefficients in some field~$E$.

If $M$ is an adjoint motive,
then $E = \mathrm{End}(M)$ is a totally real field.
Thus we could extend coefficients to the real closure of~$E$,
and get a nice decomposition of~$M$.

$\bullet$\quad How does this change of coefficients interact
with the good ordinary reduction conjecture?

\section{Introduction}

Let $K \subset \mathbb{C}$ be a finitely generated field,
and let $E$ be a totally real number field.
Let $M$ be motive (in the sense of Andr\'e)
over~$K$ with coefficients in~$E$.

Let $X$ be a model of~$K$,
and let $x \in X^\mathrm{cl}$ be a closed point.
Let $\lambda$ be a finite place of~$E$.
For the purpose of this article,
we say that $M$ has \emph{good reduction} at~$x$
if $\HH_\lambda(M)$ is unramified at~$x$
for one (and hence, every) prime~$\ell$ that is coprime to~$x$.

If $M$ has good reduction at~$x$,
we say that $M$ has \emph{ordinary} reduction
if the Newton polygon coincideds with the Hodge polygon.

\printbibliography

\end{document}

