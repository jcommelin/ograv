\documentclass{amsart}
\usepackage{amsmath,amssymb}

\usepackage{microtype}
\usepackage{geometry}
\usepackage{biblatex}
\bibliography{\jobname.bib}

\newtheorem{thm}{Theorem}[section]
\newtheorem{lem}[thm]{Lemma}
\newtheorem{prp}[thm]{Proposition}
\newtheorem{cor}[thm]{Corollary}
\newtheorem{dfn}[thm]{Definition}
\newtheorem{cnj}[thm]{Conjecture}
\newtheorem{exa}[thm]{Example}
\newtheorem{rem}[thm]{Remark}

\def\HH{\textnormal{H}}

\title{On ordinary good reduction of abelian varieties}
\author{Johan Commelin}

\begin{document}
\maketitle

\section{General plan}

Give an overview of what is known:
\cite{MR1603865},
\cite{MR3494322},
Noot?

Prove that having ordinary reduction at~$v$ only depends on the adjoint motive.
Conclude that it suffices to look at irreducible abelian motives.

Prove the good ordinary reduction conjecture
for irreducible abelian motives of type $A_n^\wedge$.

$\bullet$\quad Need to figure out how to use categories of motives
with coefficients in some field~$E$.

If $M$ is an adjoint motive,
then $E = \mathrm{End}(M)$ is a totally real field.
Thus we could extend coefficients to the real closure of~$E$,
and get a nice decomposition of~$M$.

$\bullet$\quad How does this change of coefficients interact
with the good ordinary reduction conjecture?

Let $K \subset \mathbb{C}$ be a finitely generated field,
and let $E$ be a totally real number field.
Let $M$ be motive (in the sense of Andr\'e)
over~$K$ with coefficients in~$E$.

Let $X$ be a model of~$K$,
and let $x \in X^\mathrm{cl}$ be a closed point.
For the purpose of this article,
we say that $M$ has \emph{good reduction} at~$x$
if $\HH_\lambda(M)$ is unramified at~$x$
for one (and hence, every) finite place~$\lambda$ of~$E$ that is coprime to~$x$.

If $M$ has good reduction at~$x$,
we say that $M$ has \emph{ordinary} reduction
if the Newton polygon coincides with the Hodge polygon.

Claim: These polygons only depend on~$M$ and~$x$.
Not on any choice of~$\lambda$.

Claim: If $E'$ is a totally real overfield of~$E$,
then $M \otimes_E E'$ is a motive with coefficients in~$E'$,
and $M$ has ordinary reduction at~$x$
if and only if $M \otimes_E E'$ has.

Proof: Follow your nose, Luke!

\section{Introduction}

Let $C$ be an elliptic curve over~$\mathbb{Q}$.
If $p$ is a prime, and $C$ has good reduction at~$p$,
then we can distinguish two types of good reduction:
so called \emph{ordinary} and \emph{supersingular} reduction.
In the 1960s % TODO ref
Serre proved the following facts:
\begin{itemize}
 \item If $C$ does not have complex multiplication,
  then $C$ has ordinary good reduction at
  a set of primes with density~$1$.
 \item If $C$ has complex multiplication by an
  imaginary quadratic field~$E$,
  then $C$ has supersingular reduction
  at the primes that are inert in $E/\mathbb{Q}$,
  while $C$ has ordinary reduction at the primes that split.
  (Both these sets have density~$1/2$.)
\end{itemize}
This example sketches the theme of this paper:
we are concerned with the density of the set of places
where a given abelian variety has ordinary good reduction.
This question has been studied before by
Noot, Pink, Sawin, and others.

\section{Todos}

\begin{itemize}
 \item Define $K^\diamond(M)$.
 \item Remark that $\mathrm{ORC}$ is additive.
 \item Show that $\mathrm{ORC}(M)$ only depends on $M^\mathrm{ad}$.
\end{itemize}

\section{Thoughts and stuff}

\begin{cnj}
 Let $K \subset \mathbb{C}$ be a finitely generated field,
 and let $E$ be a totally real number field.
 Let $M \in \mathrm{Mot}_K(E)$ be a motive.
 Assume that $K = K^\diamond(M)$.
 We denote with $\mathrm{ORC}(M)$ the conjecture:
 \[
  \bigl\{ x \in X^\mathrm{cl} \bigm|
  M~\text{has ordinary reduction at~$x$}\bigr\} \qquad\text{has density~$1$}
 \]
\end{cnj}

\begin{thm}
 Let $K \subset \mathbb{C}$ be a finitely generated field,
 and let $E$ be a totally real number field.
 Let $M \in \mathrm{Mot}_K(E)$ be an irreducible adjoint abelian motive.
 Assume that $\mathrm{End}(M) = E$ and $K = K^\diamond(M)$.
 If $M$ is of type $A_n^\wedge$ then $\mathrm{ORC}(M)$ holds.
\end{thm}
\begin{proof}
 HELP. TODO!
\end{proof}

\bigskip\noindent
\emph{What might happen if you extend coeffients of a motive/HS/representation.}\quad
Suppose that you have a HS~$V$, of abelian type.
Let's also assume that it is adjoint, and irreducible.
Let $E = \mathrm{End}(V)$ be the endomorphism algebra.
For simplicity, we assume that $E$ is Galois over~$\mathbb{Q}$.
Then we can look at $V \otimes_{\mathbb{Q}} E$.
Now we get a Hodge structure over~$E$, that splits into $\deg(E)$ pieces.
According to the embeddings of~$E$ into~$\mathbb{C}$
that are \emph{compact} or \emph{non-compact} (see Deligne)
we get pieces that are trivial or non-trivial.
For the trivial pieces the endomorphism algebra increases,
and the Mumford--Tate group decreases!
This is not pathological.
It just means that over~$E$ there is a smaller Mumford--Tate group,
but that one couldn't be defined over~$\mathbb{Q}$.

\printbibliography

\end{document}

